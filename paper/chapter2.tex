
\chapter{Theoretische Grundlagen}

In diesem Kapitel werden die theoretischen Grundlagen der beiden Verifikationswerkzeuge kurz erläutert. Dabei wird bewusst
auf ausführliche Formalismen verzichtet, diese können in den Quellen bei Bedarf nachgelesen werden.

Als erstes wird das Hoare-Kalkül beschrieben - die Grundlage der Softwareverifikation von Frama-C als auch VeriFast.
Anschließend werden die Begriffe der partiellen und totalen Korrekheit im Zusammenhang mit der Terminierung eines Programms
eingeführt. Der letzte Abschnitt widmet sich der Separierungslogik - der theoretischen Basis für die Verifizierung
mit VeriFast.

\section{Hoare-Kalkül}

Das Hoare-Kalkül ist ein formales System zum Beweisen der Korrektheit von Programmen. Es verwendet eine Reihe logischer
Regeln und Axiome, die direkt auf den Quellcode angewendet werden und mit sogenannten Hoare-Tripeln arbeiten. Ein solches
Tripel beschreibt den Zustand vor und nach der Ausführung eines Programmteils mit Hilfe von prädikatenlogischen
Formeln.
\begin{displaymath}
\{P\} \: S \: \{Q\}
\end{displaymath}
Die Formel P gilt vor, Q hingegen nach der Ausführung von S. Damit kann nun die Ableitungsregel zur Komposition von
zwei aufeinander folgenden Programmabschnitten aufgestellt werden:
\begin{displaymath}
\frac{\{P\} \:S\: \{R\} \:, \: \{R\} \: T \: \{Q\}}{\{P\}\: S; T \: \{Q\}}
\end{displaymath}
Die Regeln besagen, dass die unter dem Strich stehende Aussage aus der über dem Strich notierten Aussage folgt. In diesem
Fall beschreibt die Regel die Verkettung zweier Tripel.

Da die im Fokus stehenden Algorithmen mit Schleifen arbeiten, sei hier noch die Iterationsregel gezeigt, die auf While-Schleifen
bzw. auf For-Schleifen im C-Code angewendet wird: 
\begin{displaymath}
\frac{\{I \land B\} \:S\: \{I\}}{\{I\}\: while(B)\: S\: \{I \land \neg B\}}
\end{displaymath}
I bezeichnet die Schleifeninvariante, S den Schleifenkörper und B die Eintrittsbedingung. Kann nachgewiesen werden,
dass S die Invariante erhält und vor der Ausführung B und I wahr sind, dann gilt die unter dem Strich stehende Aussage: Nachdem
die Schleife durchlaufen ist, gilt nicht B und weiterhin die Invariante. B und S sind direkt aus dem Quellcode entnommen, 
die Schleifeninvariante hingegen muss bei der Verifizierung  manuell angegeben werden (siehe dazu den Abschnitt über 
Schleifeninvarianten im Kapitel 4).

Für die Verifizierung müssen nun alle Regeln auf den Quellcode angewendet werden. Eine von Frama-C verwendete Variation des Hoare-Kalküls
stellt das System der schwächsten Vorbedingungen (engl. weakest preconditions) dar. Dabei findet eine Rückwärtsanalyse
des Quellcodes statt. Begonnen wird mit den zu beweisenden Nachbedingungen. In jedem Schritt wird dann die schwächste Vorbedingung
abgeleitet, sodass am Ende zu beweisen ist, dass die zuletzt berechnete Vorbedingung aus der Vorbedingung des Kontrakts folgt.
Ist dieses Vorgehen erfolgreich, so erfüllt der Quellcode die gegebene Spezifikation aus Vor- und Nachbedingung.

\section{Terminierung}

Mit Hilfe des Hoare-Kalküls lässt sich nicht nachweisen, dass ein Programm terminiert, d.h. nach endlich vielen Schritten 
beendet ist. Man spricht deshalb von der partiellen Korrektheit, da das Programm nur dann das gewünschte 
Verhalten zeigt, wenn es auch tatsächlich zum Ende kommt.

Die Iterationsregel lässt sich z.B. auch dann anwenden, wenn es sich um eine Endlosschleife handelt. Der Nachweis,
dass die Schleife tatsächlich terminiert, ist zusätzlich zu erbringen. 

Die folgende erweiterte Iterationsregel erbringt bei Anwendung genau diesen Beweis. Für die entsprechende Schleife ist damit, die
sogenannte totale Korrektheit gezeigt:
\begin{displaymath}
\frac{\{I \land B \land t = z \} \:S\: \{I \land t < z\}, I \implies t \geq 0}{\{I\}\: while(B)\: S\: \{I \land \neg B\}}
\end{displaymath}

Diese erweiterte Regel ergänzt die als \lstinline{t} bezeichnete Schleifenvariante. Durch die zusätzliche Bedingung, dass aus der gültigen Invariante
eine positive Variante folgt, ergibt sich die Terminierung der Schleife. Denn in jedem Schleifendurchlauf verringert sich die Variante,
sodass letztendlich irgendwann ein Ende erreicht sein muss\footnote{Die Variante t könnte in diesem Fall mit natürlichen Zahlen arbeiten,
das Prinzip ist aber nicht darauf beschränkt, siehe dazu die formalen Ausführungen in \cite{floyd}[Seite 31].}.

Der Nachweis der totalen Korrektheit wird in dieser Arbeit für die vorgestellten einfachen Algorithmen angestrebt.
Allerdings ist es nicht möglich die Terminierung für beliebige Algorithmen zu beweisen, da es sich um ein nicht entscheidbares 
Problem handelt\footnote{Der Mathematiker Alan Turing hat bereits 1936 bewiesen, dass es keinen Algorithmus gibt, der 
entscheiden kann, dass ein beliebiger Algorithmus zum Ende kommt\cite{turing}[Seite 230-265].}.

\section{Separierungslogik}
\label{sec:theorie:seperation-logic}

Die Separierungslogik ist eine Erweiterung des Hoare-Kalküls, die weitergehende Aussagen über den Speicherinhalt
und den Zugriff darauf erlaubt. Die Bedeutung des Hoare-Tripels wird dazu etwas erweitert - der Programmcode des
betrachteten Tripels darf nur noch auf den Speicher zugreifen, der in der Vorbedingung erwähnt oder aber im Code
allokiert wurde\cite{reynolds-2002}[Kapitel 4].

Außerdem wird die von Hoare definierte zugrunde liegende Programmiersprache erweitert, um Befehle zum Anfordern (engl. allocate), 
Löschen, Manipulieren und Auslesen des Speichers. Der Ausführungszustand wird um zwei neue Komponenten erweitert, damit 
diese Aktionen abgebildet werden können: Der dynamische Speicherbereich (engl. heap) verknüpft Adressen mit ihren
Werten; die lokalen Variablen (engl. store/stack) assoziieren die Namen der Variablen mit ihren Inhalten. 

Die neu eingeführten Konstrukte sind bewusst stark angelehnt an die maschinennahen Konzepte der Zeigerarithmetik sowie des
Stapelspeichers. Somit ist die Anwendung der Logik auf C-Programme leicht möglich.

Damit Vor- und Nachbedingungen Aussagen über den Speicher treffen können, erweitert die Separierungslogik die
von Hoare verwendete Prädikatenlogik um folgende Operatoren bzw. Atome: 
\begin{align*}
\langle Aussage \rangle & ::= \dots \\
& \begin{array}{l r}
  | \: \textbf{emp} & \textrm{leerer Speicher}\\
  | \: \langle Ausdruck \rangle \to \langle Ausdruck \rangle & \textrm{einzelner Wert im Speicher}\\
  | \: \langle Aussage \rangle \ast \langle Aussage \rangle & \textrm{disjunkte Speicherbereiche, Konjunktion}\\
  | \: \langle Aussage \rangle -\!\! \ast \langle Aussage \rangle & \textrm{disjunkte Speicherbereiche, Implikation}\\
\end{array}
\end{align*}
Insbesondere die spezielle Konjunktion ist an dieser Stelle wichtig, da sie auch in VeriFast eine wichtige Rolle spielt.
Sie stellt sicher, dass beide Aussagen zutreffen und die entsprechenden Speicherbereiche disjunkt sind. 

Das folgende Hoare-Tripel zeigt ein Beispiel für die Anwendung der Operatoren. Der gezeigte Programmabschnitt ist in C-Syntax
formuliert und demonstriert die Semantik des Konjunktions-Operators.
\begin{align*}
& \{a \to 5 \ast b \to 5 \} \\
& \text{*a = *a + *b;} \\
& \{a \to 10 \ast b \to 5\}
\end{align*}
Dieses Tripel wäre ungültig, wenn man eine einfache logische Konjunktion wie aus der Aussagenlogik einsetzen würde. Dann
wäre nicht auszuschließen, dass die Variablen \lstinline{a} und \lstinline{b} dieselbe Speicheradresse meinen.

VeriFast unterstützt außerdem das Konzept der geteilten Speicherzugriffe (engl. fractional persmissions). Das
ist eine Ergänzung zur Separierungslogik mit der das Aufsplitten von Lesezugriffen für Speicherbereiche möglich ist\cite{concurrent}.
Damit ist auch das Formalisieren und Beweisen paralleler Programme möglich.  

Der Funktionsumfang der Annotationssprache für VeriFast ist also nicht deckungsgleich mit dem aus der Arbeit von Reynolds.
Teilweise wurde die Sprache reduziert und an anderen Stellen erweitert, die Nutzung von Quantoren ist in VeriFast beispielsweise
nicht erlaubt.
