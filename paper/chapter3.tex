
\chapter{Grundkonzepte}

\section{Design by Contract}
\label{sec:design-by-contract}
\subsection{Grundidee}

Bei der formalen Verifikation geht es darum sicherzustellen, dass ein entsprechendes Stück Software
funktional korrekt ist. Das wird erreicht, in dem man die Bausteine der Sprache - in C sind es zum Beispiel die Methoden 
oder Strukturen -
Stück für Stück formalisiert und verifiziert. Die Signatur einer Methode stellt bereits einen einfachen Vertrag zwischen
Aufrufenden und Aufrufer dar - Parameteranzahl und ihre Typen sind festgelegt, genauso wie der Rückgabewert.
Um auch die funktionalen Aspekte der Methode zu beschreiben, wird diese Vereinbarung um sogenannte Vor- und Nachbedingungen erweitert.
Diese beschreiben das Resultat der Methode sowie notwendige Annahmen, die vor dem Aufrufen der Methode gegeben sein müssen. 

Somit entstehen exakte vollständig formalisierte Bausteine, die keinen Raum für Ungenauigkeiten oder Interpretationen zulassen.
Diese Herangehensweise nennt sich Design by Contract und ist die Grundidee der meisten Verifikationswerkzeuge.



\subsection{Einfache ACSL-Spezifikation}
\label{sec:design-by-contract:acsl-spezifikation}

Nachfolgend ist ein einfaches Beispiel einer ACSL-Funktions-Spezifikation dargestellt.
Die Funktion equal prüft zwei Ganzzahlen-Arrays auf Gleichheit:

\lstinputlisting[language=C]{codes/equal_contract_acsl.c}

Der spezielle @-Präfix in Zeile 1 signalisiert dem Verifikationswerkzeug, 
dass der folgende Kommentarblock als Verifikations-Annotation zu interpretieren ist. Normale C-Compiler hingegen
ignorieren den Kommentar.

\todo{gab es in acsl nicht auch assumes?}

Die Vorbedingungen sind in Zeile 2 und 3 zu finden: \lstinline{IsValidRange} ist ein Prädikat, welches hier sicherstellt,
dass die Zeiger \lstinline{a} und \lstinline{b} gültig sind und auf den Speicherbereich \lstinline{a[0..n]} 
(von \lstinline{a[0]} bis \lstinline{a[n-1]}) bzw. \lstinline{b[0..n]} zugegriffen 
werden darf. Zusätzlich verbietet das Prädikat negative Array-Größen.

Die \lstinline{\assigns}-Klausel beschreibt potenzielle Seiteneffekte. Da es sich bei equal um eine nicht mutierende
Funktion handelt, wird hier kein Speicher verändert.

Der Rückgabewert, identifiziert durch \lstinline{\result}, wird durch die Nachbedingung in Zeile 7 definiert. Er ist genau dann
wahr, wenn die Ganzzahlen-Arrays \lstinline{a[0..n]} und \lstinline{b[0..n]} elementweise gleich sind. Ansonsten ist der 
zurückgegebene Bool-Wert falsch. 

Erreicht wird dies mit Hilfe des Prädikats \lstinline{IsEqual}. Dieses wiederum
nutzt folgende Aussage in Prädikatenlogik (erster Stufe mit Identifikation), um zu definieren wann genau 
zwei Arrays \lstinline{a} und \lstinline{b} der Länge \lstinline{n} gleich sind:
\[IsEqual(a, n, b) \equiv \forall(i) (0 \leq i < n \rightarrow a[i] = b[i])\]

Weitere Details zur allgemeinen Funktionsweise als auch speziell zu diesen Prädikaten siehe
\ref{sec:design-by-contract:predicates}.



\subsection{Variante mit Verifast}
\label{sec:design-by-contract:verifast-variante}

Als Vergleich nun eine semantisch identische Spezifikation in Verifast:

\lstinputlisting[language=C]{codes/equal_contract_verifast.c}

Als erstes fällt die unterschiedliche Platzierung der Annotation auf: In Verifast werden die 
Annotations-Kommentare nach der Signatur eingefügt. Außerdem müssen alle Vor- und Nachbedingungen
in nur einer \lstinline{requires} bzw. \lstinline{ensures}-Anweisung zusammengefasst werden.

Eine \lstinline{assigns}-Klausel gibt es zudem auch nicht. Dennoch ist zu erkennen, dass es sich um einen 
nicht-mutierenden Algorithmus handelt. Anders als in ACSL wird nicht explizit angegeben ob und auf welche 
Speicherstellen zugegriffen werden darf. Stattdessen wird eine Aussage über den Speicherinhalt selbst getroffen.

Um Aussagen über den Speicherinhalt mit logischen Aussagen zu verbinden, gibt es in Verifast die spezielle 
Konjunktion \lstinline{&*&}. In diesem Beispiel wird hier z.B. die Vorbedingung \lstinline{n >= 0} mit Aussagen
zu den Speicherbereichen \lstinline{a[0..n]} und \lstinline{b[0..n]} verknüpft.

Der \lstinline{x |-> y} Operator bedeutet dabei, dass die Speicherstelle \lstinline{x} den Inhalt \lstinline{y} hat.
Unbekannte Werte werden dabei über eine Mustererkennung (engl. pattern matching) ergänzt. Die Klausel
\lstinline{a[0..n] |-> ?al} hat damit zwei Aufgaben: Der erste Teil stellt sicher, dass der Zeiger a und
der dahinterliegende Speicherbereich \lstinline{a[0..n]} gültig ist; der zweite identifiziert eben diesen 
Speicherbereich und bindet ihn an die Variable \lstinline{al}.
Verifast erlaubt es nun im Gegensatz zu ACSL auf diese Variablen auch in den Nachbedingungen zuzugreifen. Damit
hat man nun ein Mittel, mit dem man Seiteneffekte exakt beschreiben sowie Speicherbereiche vergleichen kann.

Die Klauseln \lstinline{a[0..n] |-> al} und \lstinline{b[0..n] |-> bl} in Zeile 3 stellen nun sicher, dass der
Speicher unverändert bleibt. Der Rückgabewert wird in Verifast mit \lstinline{result} (ohne vorangestellten 
Backslash wie in ACSL) bezeichnet. Der Ausdruck \lstinline{result == (al == bl)} bedeutet also, dass der von der Funktion
zurückgegebene Wert genau dann wahr ist, wenn die Speicherbereiche \lstinline{al} und \lstinline{bl} gleich sind.

Die Verifast-Spezifikation hat also tatsächlich die gleiche Semantik, erreicht diese aber auf eine ganz andere
Art und Weise.



\subsection{Einfache und rekursive Prädikate}
\label{sec:design-by-contract:predicates}

Prädikate sind ein grundlegendes Abstraktionsmittel in ACSL und Verifast. Sie ermöglichen die Wiederverwendung
von Spezifikations-Teilen und nicht zuletzt verbessern sie auch wesentlich die Lesbarkeit.

Weiter oben wurde das Prädikat \lstinline{IsEqual} definiert, hier sei nun exemplarisch die ACSL-Notation
für diese Aussage dargestellt:

\lstinputlisting[language=C]{codes/equal_predicate_acsl.c}

Die Annotation ist eine direkte Umsetzung der Formel von oben - die Interpretation ist klar verständlich für 
jeden, der prädikatenlogische Formeln lesen kann. Erwähnenswert ist noch die Verwendung des generischen Typen
\lstinline{value_type}, der das Prädikat auch für anderen primitive Typen wie \lstinline{long} oder
\lstinline{short} anwendbar macht.

\todo{in acsl gibt es induktive prädikate, die ähnliches erlauben}
In Verifast sind neben solch einfachen Prädikaten auch rekursive möglich. Damit ergibt sich die Möglichkeit
sogar unbegrenzte Datenstrukturen zu beschreiben \todo{referenz auf verifast-tutorial seite 9, chapter 7}. In dieser
Arbeit sind allerdings nur Arrays mit bekannter Länge Untersuchungsgegenstand. Doch auch für diese sind
rekursive Prädikate in Verifast die natürliche Art der Beschreibung. Beispielsweise können Ganzzahl-Arrays
wie folgt definiert werden:

\lstinputlisting[language=C]{codes/int_array_verifast.c}
 
Das Prädikat verweist solange auf sich selbst bis \lstinline{count <= 0}. Dass jede Speicherstelle tatsächlich
gültig ist, stellt das in Verifast eingebaute Prädikat \lstinline{integer} sicher. In dieser Form ist das Prädikat
\lstinline{int_array} nun praktisch identisch mit der \lstinline{IsValidRange}-Variante aus der ACSL-Spezifikation, 
die in \ref{sec:design-by-contract:acsl-spezifikation} zu sehen war.

Auffällig ist an der Stelle noch die Verwendung des \lstinline{_}-Zeichens in Zeile 3. Wie in vielen anderen 
Sprachen auch bezeichnet es eine anonyme Variable, die an dieser Stelle zwar angegeben werden muss, aber deren 
Inhalt nicht weiter interessiert. 

Wird das \lstinline{integer} Prädikat ohne anonyme Variable verwendet, bedeutet der Aufruf von 
\lstinline{integer(start, ?s)} so viel wie \lstinline{*start |-> ?s}. Allerdings erlaubt Verifast
den Einsatz des \lstinline{|->} Operators nicht zusammen mit der Derefenzierung eines einzelnen
Ganzzahl-Wertes.

Tatsächlich ist der Einsatz des \lstinline{int_array(a, n)} Prädikats äquivalent mit der Schreibweise 
von \lstinline{a[0..n] |-> _}, denn Verifast behandelt die Array-Notation intern wie ein rekursives
Prädikat.

Eine erweiterte Fassung dieses Prädikats wird weiter unten in \todo{referenz detaillierter auflösen}
\ref{sec:abstrakte-datentypen} beschrieben.



\subsection{Partielle Spezifikationen}
\label{sec:design-by-contract:partielle-korrektheit}

Das Ziel der in \ref{sec:design-by-contract:acsl-spezifikation} vorgestellten Spezifikation war die
vollständige funktionale Definition, damit kein Raum für Interpretationen frei bleibt. Allerdings ist
das nicht immer gewünscht oder gar nicht das Ziel - für das Aufspüren von Null-Zeiger-Ausnahmen 
oder Speicherlöchern z.B. reicht ein eingeschränkter Funktionsvertrag. Funktionale Aspekte werden dafür 
weggelassen oder ggf. vereinfacht:

\lstinputlisting[language=C]{codes/equal_partial_contract_verifast.c}
 
Diese Spezifikation ist immer noch korrekt für eine equal-Implementierung, stellt aber selber nur sicher,
dass die Zeiger gültig sind und auf den Speicherbereich zugegriffen werden darf. Der Rückgabewert sowie 
der nicht mutierender Charakter des Algorithmus ist nun nicht mehr formalisiert. 

Denkbar ist es auch nur die mögliche Aufrufreihenfolge (einer gedachten Zustandsmaschine) zu verifizieren,
ohne alle weiteren Nachbedingungen und Ergebnisse zu formalisieren.



\section{Abstrakte Datentypen}
\label{sec:abstrakte-datentypen}

induktive datentypen an hand von liste zeigen
recursive logic definitions in acsl
in verifast sind listen aber wichtiger, da es keine quantoren gibt
zudem ist es einfacher mit abstrakten datentypen zu arbeiten, da für diese viele (bereits bewiesene)
hilfsfunktionen existieren
in acsl gibt es auch typen (in verifast wären das prädikate?) .. hier nicht genauer erklärt 

zeigen dass a[0..n] |-> ?al andere schreibweise für rekursives prädikat ints(a, n, ?al) ist

generics auch hier bei listen in verifast

keine entsprechung in frama-c (beschränktheit prädikatenlogik erster stufe - liste von listen?)


kombination mit rekursiven prädikaten erläutern


mismatch-spezifikation zeigen


\section{Behaviors}


behaviors drücken pre- und postconditions in frama-c lesbarer aus


keine alternative dafür in verifast, dafür kann manches kürzer ausgedrückt werden mit hilfe von induktiven datentypen


\section{Verifikation von Implementierungen}

Rekursive Implementierung von mismatch
in verifast

ghost commands
(prädikate öffnen)
(präzise prädikate)


produce/consume assertions
toolunterstützung verifast zeigen und erklären


in acsl zeigen
unterschied erklären


schleifeninvarianten


