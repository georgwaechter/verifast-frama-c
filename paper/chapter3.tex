
\chapter{Grundkonzepte}

\section{Design by Contract}
\label{sec:design-by-contract}

spezifikation für methoden 


pre und post condition, abstrakte datentypen (siehe 2.)
  
  
pre condition beschreibt annahmen für die ausführung
  
  
post condition beschreibt effekt der methode 
  (rückgabewert und ggf. seiteneffekte)

  
partielle korrektheit zeigen (keine funktionale), gültige zeiger und kein speicherleck
(prädikate, rekursives prädikat für speicherbereich)
  
funktionale korrektzeit spezifizieren: einfaches equal beispiel 
(vorher prädikatenlogik-variante ggf)
(acls-variante zeigen)


assertions in seperation logic oder first order logic


\section{Abstrakte Datentypen}

induktive datentypen an hand von liste zeigen


keine entsprechung in frama-c (beschränktheit prädikatenlogik erster stufe - bäume?)


kombination mit rekursiven prädikaten erläutern


mismatch-spezifikation zeigen


\section{Behaviors}


behaviors drücken pre- und postconditions in frama-c lesbarer aus


keine alternative dafür in verifast, dafür kann manches kürzer ausgedrückt werden mit hilfe von induktiven datentypen


\section{Verifikation von Implementierungen}

Rekursive Implementierung von mismatch
in verifast

ghost commands
(prädikate öffnen)
(präzise prädikate)


produce/consume assertions
toolunterstützung verifast zeigen und erklären


in acsl zeigen
unterschied erklären


schleifeninvarianten


