\chapter{Vergleich von Spezifikationen}
\label{sec:design-by-contract}

Bei der formalen Verifikation geht es darum sicherzustellen, dass ein entsprechendes Stück Software
funktional korrekt ist. Das wird erreicht, in dem man die Bausteine der Sprache - in C sind es zum Beispiel die Methoden 
oder Strukturen -
Stück für Stück formalisiert und verifiziert. Die Signatur einer Methode stellt bereits einen einfachen Vertrag zwischen
Aufrufenden und Aufrufer dar - Parameteranzahl und ihre Typen sind festgelegt, genauso wie der Rückgabewert.
Um auch die funktionalen Aspekte der Methode zu beschreiben, wird diese Vereinbarung um sogenannte Vor- und Nachbedingungen erweitert.
Diese beschreiben das Resultat der Methode sowie notwendige Annahmen, die vor dem Aufrufen der Methode gegeben sein müssen. 

Somit entstehen exakte vollständig formalisierte Bausteine, die keinen Raum für Ungenauigkeiten oder Interpretationen zulassen.
Diese Herangehensweise nennt sich Design by Contract und ist die Grundidee der meisten Verifikationswerkzeuge.


\section{ACSL-Spezifikation für \texttt{equal}}
\label{sec:design-by-contract:acsl-spezifikation}

Nachfolgend ist ein einfaches Beispiel einer ACSL-Funktions-Spezifikation dargestellt.
Die Funktion equal prüft zwei Ganzzahlen-Arrays auf Gleichheit:

\lstinputlisting[language=C, caption=Formale Spezifikation der Funktion \lstinline{equal} mit ACSL]{codes/equal_contract_acsl.c}

Der spezielle @-Präfix in Zeile 1 signalisiert dem Verifikationswerkzeug, 
dass der folgende Kommentarblock als Verifikations-Annotation zu interpretieren ist. Normale C-Compiler hingegen
ignorieren den Kommentar.

Die Vorbedingungen sind in Zeile 2 und 3 zu finden: \lstinline{IsValidRange} ist ein Prädikat, welches hier sicherstellt,
dass die Zeiger \lstinline{a} und \lstinline{b} gültig sind und auf den Speicherbereich \lstinline{a[0..n]} 
(von \lstinline{a[0]} bis \lstinline{a[n-1]}) bzw. \lstinline{b[0..n]} zugegriffen 
werden darf. Zusätzlich verbietet das Prädikat negative Array-Größen.

Die \lstinline{\assigns}-Klausel beschreibt potenzielle Seiteneffekte. Da es sich bei equal um eine nicht mutierende
Funktion handelt, wird kein Speicherinhalt (\lstinline{\nothing}) verändert.

Der Rückgabewert, identifiziert durch \lstinline{\result}, wird durch die Nachbedingung in Zeile 7 definiert. Er ist genau dann
wahr, wenn die Ganzzahlen-Arrays \lstinline{a[0..n]} und \lstinline{b[0..n]} elementweise gleich sind. Ansonsten ist der 
zurückgegebene Bool-Wert falsch. 

Erreicht wird dies mit Hilfe des Prädikats \lstinline{IsEqual}. Dieses wiederum
nutzt folgende Aussage in Prädikatenlogik (erster Stufe mit Identität), um zu definieren wann genau 
zwei Arrays \lstinline{a} und \lstinline{b} der Länge \lstinline{n} gleich sind:
\[IsEqual(a, n, b) \equiv \forall(i) (0 \leq i < n \rightarrow a[i] = b[i])\]

Weitere Details zur allgemeinen Funktionsweise als auch speziell zu diesen Prädikaten siehe
\ref{sec:design-by-contract:predicates}.



\section{Verifast-Spezifikation für \texttt{equal}}
\label{sec:design-by-contract:verifast-variante}

Als Vergleich nun eine semantisch identische Spezifikation in Verifast:

\lstinputlisting[language=C, caption=Formale Spezifikation der Funktion \lstinline{equal} mit Verifast]{codes/equal_contract_verifast.c}

Als erstes fällt die unterschiedliche Platzierung der Annotation auf: In Verifast werden die 
Annotations-Kommentare nach der Signatur augeführt und nicht davor. Außerdem müssen alle Vor- und 
Nachbedingungen in nur einer einzigen \lstinline{requires} bzw. \lstinline{ensures}-Anweisung 
zusammengefasst werden.

Eine \lstinline{assigns}-Klausel taucht auch nicht auf. Dennoch ist erkennbar, dass es sich um einen 
nicht-mutierenden Algorithmus handelt. Anders als in ACSL wird nicht explizit angegeben ob und auf welche 
Speicherstellen zugegriffen werden darf. Stattdessen wird eine Aussage über den Speicherinhalt selbst 
getroffen. Der \lstinline{x |-> y} Operator (engl. points-to) spielt dabei eine wichtige Rolle. Er gibt an, 
dass die Speicherstelle \lstinline{x} den Inhalt \lstinline{y} hat. Gleichzeitig stellt Verifast sicher, 
dass der Zeiger \lstinline{x} gültig ist - analog zur \lstinline{\valid(x)} Anweisung aus ACSL.

Unbekannte Werte werden per Mustererkennung (engl. pattern matching) ergänzt. Die Klausel
\lstinline{a[0..n] |-> ?al} hat damit zwei Aufgaben: Sicherstellen, dass die Zeiger \lstinline{a[0..n]}
gültig sind; sowie den dahinterliegenden Speicherinhalt identifizieren und an die Variable \lstinline{al}
binden. Verifast erlaubt die Verwendung dieser Variablen sogar in den Nachbedingungen. Damit hat man ein 
adäquates Mittel, um alte Werte (aus der Zeit des Methodenaufrufes) zu verwenden, um Seiteneffekte zu
beschreiben sowie Speicherinhalte zu vergleichen. In ACSL wäre für eine ähnliche Herangehensweise
die Verwendung von \lstinline{\old} notwendig.

Aussagen über den Speicherinhalt werden in Verifast mit der speziellen Konjunktion \lstinline{&*&} 
verbunden. In diesem Beispiel wird hier z.B. die Vorbedingung \lstinline{n >= 0} mit Aussagen zu den 
Speicherbereichen \lstinline{a[0..n]} und \lstinline{b[0..n]} verknüpft. Der \lstinline{&&} Operator 
existiert zusätzlich für logische Verknüpfungen.

Die Klauseln \lstinline{a[0..n] |-> al} und \lstinline{b[0..n] |-> bl} in Zeile 3 stellen nun also sicher, 
dass der Speicherinhalt unverändert bleibt. Der Rückgabewert wird in Verifast mit \lstinline{result} 
(ohne vorangestellten Backslash wie in ACSL) bezeichnet. Der Ausdruck \lstinline{result == (al == bl)} 
bewirkt, dass der von der Funktion zurückgegebene Wert genau dann wahr ist, wenn die Inhalte der 
Speicherbereiche \lstinline{al} und \lstinline{bl} gleich sind.

Die Verifast-Spezifikation hat also tatsächlich die gleiche Semantik, erreicht dies aber wegen der 
zu Grunde liegenden Seperation Logic (siehe \ref{sec:theorie:seperation-logic}) auf eine andere ganz 
andere Art und Weise.
\newline
\newline
Wichtig zu erwähnen ist noch, dass die \lstinline{\assigns nothing}-Klausel aus ACSL genau wie Verifast 
theoretisch das Ändern des Speicherinhalts erlaubt, wenn er zum Ende der Funktion wieder rekonstruiert wird.
Denn es wird lediglich überprüft, dass der Inhalt nach dem Aufruf mit dem vor
dem Aufruf übereinstimmt. Eine Feinheit, die im Kontext der Parallelprogrammierung wichtig ist, wenn
die Funktion inmitten der Ausführung unterbrochen wird bzw. parallel zu einer anderen ausgeführt wird,
die auf gemeinsame Variablen zurückgreift.

Um die Manipulation zu jeder Zeit zu verbieten, hat Verifast das Konzept der sogenannten Fractional Permissions.
Damit lässt sich steuern, ob eine Funktion nur lesend oder auch schreibend auf Speicherbereiche zugreifen darf.
Auf die Verwendung dieses Sprachmittels wird in dieser Arbeit jedoch nicht weiter eingegangen.



\section{Einfache und rekursive Prädikate}
\label{sec:design-by-contract:predicates}

Prädikate sind ein grundlegendes Abstraktionsmittel zum Zusammenfassen und Wiederverwenden von logischen 
Formeln in ACSL und Verifast.

Weiter oben wurde das Prädikat \lstinline{IsEqual} definiert, hier sei nun exemplarisch die ACSL-Notation
für diese Aussage dargestellt:

\begin{figure}[H]
\lstinputlisting[language=C, caption=Prädikat \lstinline{IsEqual} formuliert in ACSL]{codes/equal_predicate_acsl.c}
\end{figure}

Die Annotation ist eine direkte Umsetzung der Formel von oben - die Interpretation ist klar verständlich für 
jeden, der prädikatenlogische Formeln lesen kann.

\todo{in acsl gibt es induktive prädikate, die ähnliches erlauben}
In Verifast sind neben solch einfachen Prädikaten auch rekursive möglich. Damit ergibt sich die Möglichkeit
unbegrenzte Datenstrukturen zu beschreiben \todo{referenz auf verifast-tutorial seite 9, chapter 7}. In dieser
Arbeit sind allerdings nur Arrays mit bekannter Länge Untersuchungsgegenstand. Doch auch für diese sind
rekursive Prädikate in Verifast die natürliche Art der Beschreibung. Beispielsweise können Ganzzahl-Arrays
wie folgt definiert werden:

\begin{figure}[H]
\lstinputlisting[language=C, caption=Prädikat \lstinline{int_array} formuliert mit Verifast]{codes/int_array_verifast.c}
\end{figure}

Das Prädikat verweist solange auf sich selbst bis \lstinline{count <= 0}. Dass jede Speicherstelle tatsächlich
gültig ist, stellt das in Verifast eingebaute Prädikat \lstinline{integer} sicher. In dieser Form ist das Prädikat
\lstinline{int_array} nun praktisch identisch mit der \lstinline{IsValidRange}-Variante aus der ACSL-Spezifikation, 
die in \ref{sec:design-by-contract:acsl-spezifikation} zu sehen war.

Auffällig ist an der Stelle noch die Verwendung des \lstinline{_}-Zeichens in Zeile 3. Wie in vielen anderen 
Sprachen auch bezeichnet es eine anonyme Variable, die an dieser Stelle zwar angegeben werden muss, aber deren 
Inhalt nicht weiter interessiert. 

Wird das \lstinline{integer} Prädikat ohne anonyme Variable verwendet, bedeutet der Aufruf von 
\lstinline{integer(start, ?s)} so viel wie \lstinline{*start |-> ?s}. Allerdings erlaubt Verifast
den Einsatz des \lstinline{|->} Operators nicht zusammen mit der Derefenzierung eines einzelnen
Ganzzahl-Wertes.

Tatsächlich ist der Einsatz des \lstinline{int_array(a, n)} Prädikats äquivalent mit der Schreibweise 
von \lstinline{a[0..n] |-> _}, denn Verifast behandelt die Array-Notation intern wie ein rekursives
Prädikat.

Die vollständige Fassung dieses Prädikats wird weiter unten in \ref{sec:induktive-listen} beschrieben.



\section{Unvollständige Spezifikationen: Robustheit als Ziel}
\label{sec:design-by-contract:partielle-korrektheit}

Das Ziel der in \ref{sec:design-by-contract:acsl-spezifikation} vorgestellten Spezifikation war die
vollständige funktionale Definition, damit kein Raum für Interpretationen frei bleibt. Allerdings ist
das nicht immer gewünscht: Soll die Robustheit der Software sichergestellt werden, so ist das Aufspüren 
von potenziellen Null-Zeiger-Ausnahmen oder Speicherlöchern das Ziel und nicht die funktionale
Korrektheit. Für ersteres reicht eine reduzierte Spezifikation - funktionale Aspekte werden dabei 
weggelassen oder ggf. vereinfacht:

\lstinputlisting[language=C, caption=Unvollständige Spezifikation]{codes/equal_partial_contract_verifast.c}
 
Diese Spezifikation ist immer noch korrekt für eine equal-Implementierung, stellt aber selber nur sicher,
dass die Zeiger gültig sind und auf den Speicherbereich zugegriffen werden darf. Der Rückgabewert sowie 
der nicht mutierender Charakter des Algorithmus ist nun nicht mehr formalisiert.
\newline
\newline
Denkbar ist auch die Verifizierung der gültigen Aufrufreihenfolge (einer gedachten Zustandsmaschine),
ohne alle weiteren Nachbedingungen und Ergebnisse zu formalisieren. Somit kann sichergestellt werden,
dass sich die Zustandsmaschine niemals in einem ungültigen Zustand befindet. Ansonsten erforderliche
Laufzeit-Überprüfungen können wegfallen.



\section{Lesbarkeit von Spezifikationen}
\label{sec:design-by-contract:behaviors}

In ACSL gibt es die Möglichkeit mehrere (meist disjunkte) Fälle innerhalb einer Spezifikation getrennt
aufzuschreiben. Das erleichtert die Lesbarkeit erheblich, da man für jeden Fall die Vor- und Nachbedingungen
einzeln lesen kann.

Folgende Spezifikation von equal ist eine Variante in ACSL ohne \lstinline{IsEqual}-Prädikat:

\begin{figure}[H]
	\lstinputlisting[language=C, caption=Formale Spezifikation für \lstinline{equal} mit Behavior]{codes/equal_behavior_acsl.c}
\end{figure}

Der Fall \lstinline{all_equal} in Zeile 7 tritt dann ein, wenn beide Ganzzahl-Arrays gleich sind. Das 
Gegenteil \lstinline{some_not_equal} ist in Zeile 11 definiert, die jeweiligen Bedingungen in Form von
prädikatenlogischen Formeln finden sich als \lstinline{assumes}-Anweisungen wieder. Besonders anzumerken ist,
dass die Anweisungen \lstinline{complete behaviors} bzw. \lstinline{disjoint behaviors} dafür sorgen, dass
alle Fälle abgedeckt sowie disjunkt sind.

In Verifast ist das nur manuell zu erreichen, allerdings mit vergleichsweise längeren Ausdrücken. Reicht 
es in ACSL die \lstinline{assumes}-Fälle \lstinline{a, b, c} als \lstinline{behavior} 
aufzulisten und per \lstinline{disjoint behaviors} zu verknüpfen, so muss in Verifast die folgende
Formel geschrieben werden:
\[(a \land \neg b \land \neg c) \lor (\neg a \land b \land \neg c) \lor (\neg a \land \neg b \land c)\]

Die wichtigsten Abstraktionsmittel sind jedoch in Verifast als auch ACSL die Prädikate. 
Für die \lstinline{equal}-Spezifikation ist z.B. bei Verwendung des \lstinline{IsEqual}-Prädikats gar kein Behavior mehr notwendig,
siehe \ref{sec:design-by-contract:acsl-spezifikation}. Solch eine Spezifikation ist besser 
verständlich, nicht nur weil sie kürzer ist, auch weil die Prädikatenformel verschwunden ist. 
Damit kann sie nun mindestens oberflächlich auch dann verstanden werden, wenn kein 
prädikatenlogisches Wissen vorhanden ist.

Die Definition der möglichen Seiteneffekte ist in den Sprachen unterschiedlich gelöst. In ACSL
ist an Hand der \lstinline{assigns \nothing} Klausel sofort ersichtlich, dass keine Speichermanipulation
erfolgt. Dieser Aspekt ist in Verifast nicht so schnell zu erfassen.

\section{Induktive Listen mit Verifast}
\label{sec:induktive-listen}

Schaut man sich die Verifast-Spezifikation von \lstinline{equal} genauer an, fällt auf, dass es 
möglich ist die Inhalte der zwei Speicherbereiche mit \lstinline{==} (siehe Zeile 3) zu vergleichen. 
Dazu hier nochmal der entsprechende Code-Auszug:

\begin{figure}[H]
\lstinputlisting[language=C, caption=Formale Spezifikation für \lstinline{equal} mit Verifast]{codes/equal_contract_verifast.c}
\end{figure}

Der \lstinline{==} Operator nach der \lstinline{result}-Variable ist ein einfacher boolescher Vergleich,
der zweite \lstinline{==} Operator hingegen vergleicht die beiden Variablen \lstinline{al} und \lstinline{bl}.

Der Vergleich ist möglich, da es sich um Listen vom Typ \lstinline{list<int>} handelt. Dies ist ein
von Verifast mitgelieferter (generischer) induktiver Datentyp, ganz ähnlich zu den Listen in funktionalen 
Sprachen wie Haskell. Damit ist klar, dass die Array-Notation \lstinline{a[0..n] |-> ?al} dafür sorgt, 
dass der Speicherinhalt bei der Mustererkennung als induktive Liste gebunden wird.

Induktive Typen können wie folgt definiert werden - hier die \lstinline{list}-Definition von Verifast:

\lstinputlisting[language=C, caption=Induktive Listendefinition]{codes/inductive_list_verifast.c}

Um zu verstehen wie diese eingesetzt werden, wird das offizielle Verifast \lstinline{ints}-Prädikat genauer
erläutert. Dieses wurde oben bereits verwendet, denn es ist tatsächlich nur eine andere Schreibweise für die 
lesbarere Array-Notation. Das Binden der Inhalte aus dem Array \lstinline{a} ist darum auch alternativ via 
\lstinline{ints(a, n, ?al)} möglich.

Nachfolgend ist das rekursive \lstinline{ints}-Prädikat abgebildet. Gut zu erkennen ist dabei
wie die \lstinline{intlist} Stück für Stück rekursiv mit Hilfe der Konstruktoren
\lstinline{cons} bzw. \lstinline{nil} erzeugt wird.

\begin{figure}[H]
\lstinputlisting[language=C, caption=Prädikat \lstinline{ints}]{codes/ints_predicate_verifast.c}
\end{figure}

Mit diesen Listen kann man nun flexibel arbeiten, denn Verifast erlaubt die Definition sogenannter
\lstinline{fixpoint}-Funktionen, mit denen die Listen manipuliert werden. Die grundlegendsten
Listen-Utensilien bringt Verifast aber gleich mit, z.B. Implementierungen für \lstinline{head}, 
\lstinline{tail}, \lstinline{append} (Hinzufügen von Elementen) oder auch \lstinline{take} 
(Kopie der ersten N Elemente).



\section{Formale Spezifikation für \texttt{mismatch}}

Mit diesem Wissen ist es nun möglich eine formale Spezifikation für \lstinline{mismatch} mit Verifast
zu definieren (für das Verstehen der folgenden Erklärungen ist die informelle Spezifikation aus 
\ref{sec:aufgabenstellung} sehr hilfreich):

\begin{figure}[H]
\lstinputlisting[language=C, caption=Formale Spezifikation für \lstinline{mismatch} mit Verifast]{codes/mismatch_specification_verifast.c}
\end{figure}

Es gibt in der Nachbedingung zwei mögliche Fälle: Entweder sind die Arrays ungleich
(\lstinline{result < n}) oder sie sind gleich (\lstinline{result == n}). Das ist leider nicht direkt
in der Spezifikation wiederzufinden, denn das separate Definieren disjunkter Fällen wird in Verifast 
nicht unterstützt (siehe \ref{sec:design-by-contract:behaviors}). Die Zeile 3 kombiniert daher beide Fälle 
in einem, was die Lesbarkeit etwas beeinträchtigt - das Resultat ist dafür sehr kompakt.

Da Verifast keine Prädikatenlogik kennt, gibt es weder Quantoren noch Implikation. Am Ende der
Nachbedingung beispielsweise möchte man intuitiv \((result < n) \rightarrow al[result] \neq  bl[result]\)
schreiben, kann es aber nur wie folgt ausdrücken: 
\lstinline{result < n ? nth(result, al) != nth(result, bl) : true}.
\newline
\newline
Die ACSL-Umsetzung ist wesentlich länger und verwendet Behaviors:
\begin{figure}[H]
\lstinputlisting[language=C, caption=ACSL-Variante der Spezifikation für \lstinline{mismatch}]{codes/mismatch_specification_acsl.c}
\end{figure}

Die zwei Fälle sind in dieser Variante besser sichtbar und damit auch verständlicher, da die Spezifikation
aus kleineren Teilen besteht. Sie ist somit lesbarer, mit dem Nachteil viel Platz einzunehmen.
Die Abwägung zwischen Codelänge und Kompaktheit des Codes ist aber generell schwierig und subjektiv.

Des Weiteren fällt auf, dass die ACSL-Spezifikation direkt mit den Variablen \lstinline{a} und
\lstinline{b} arbeitet, wohingegen die Verifast-Variante die funktionalen Aspekte ausschließlich
mit Hilfe der induktiven Listen \lstinline{al} und \lstinline{bl} beschreibt. 



\section{Unterschiede in der Umsetzung}

In gewissen Grenzen wäre es zwar auch möglich die Annotationen eins zu eins zu übersetzen, aber damit
würden die Vorteile der jeweiligen Sprache verloren gehen. Ein \lstinline{IsEqual}-Prädikat wäre z.B.
auch mit Verifast möglich, macht die Spezifikation aber nicht verständlicher, sondern nur die Verifizierung
der Implementierung schwieriger. \todo{Referenz auf stelle in 3.3}

Das Angeben der Speicherinhalte und Einführen der Variablen  \lstinline{al} und \lstinline{bl} ist außerdem
ohnehin notwendig, um auszudrücken, dass der Speicher nicht verändert und nicht von der Funktion selber
gelöscht wird. Denn Verifast wertet auch \lstinline{malloc}- und \lstinline{free}-Aufrufe aus, um 
den Zugriff auf gelöschten Speicher zu verhindern als auch das doppelte Entfernen von diesem.

Es liegt also nahe diese Varaiblen auch weiter zu verwenden - die Sprachmittel von Verifast (wie die Array-Notation in Kombination mit dem
\lstinline{|->} Operator) drängen einen mehr als ACSL dazu induktive Datentypen zu nutzen.

Der Vorteil ist ein höherer Grad der Abstraktion, denn Aussagen über Listen wie z.B.
\lstinline{result < n ? nth(result, al) != nth(result, bl)} haben keine Abhängigkeit mehr zu
dem Eingabeparameter \lstinline{const int a*}. Damit hätte ein Wechsel der Datenstruktur von Arrays
hin zu verketteten Listen nur einen kleinen Einfluss auf die Spezifikation. Es wäre tatsächlich ausreichend
die Bindung der Arrays in die induktive Liste zu ändern. Statt \lstinline{a[0..n] |-> ?al} würde man also
z.B. ein Prädikat \lstinline{linked_list} verwenden: \lstinline{linked_list(a, n, ?al)}. Der Rest der 
Spezifikation kann unverändert bleiben.

\todo{mismatch spezifikation für linked lists einbinden?}

\todo{recursive logic definitions in acsl zeigen?}
\newline
\newline
Andersherrum sieht es ähnlich aus - auch in ACSL wäre es möglich induktive Datentypen zu verwenden, doch
ist das an der Stelle unnötig, da die Quantoren der Prädikatenlogik zur Beschreibung genügen. 
Der Umstieg von ACSL auf Verifast oder andersrum erfordert daher ein Umdenken, damit die Verifizierung 
nicht durch einen ineffizenten Einsatz der Sprachmittel unnötig erschwert wird.


