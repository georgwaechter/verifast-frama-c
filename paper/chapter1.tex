\chapter{Einleitung}

\section{Aufgabenstellung und Durchführung}
\label{sec:aufgabenstellung}
Diese Arbeit vergleicht die statischen Verifikationswerkzeuge VeriFast\footnote{
\url{http://people.cs.kuleuven.be/~bart.jacobs/VeriFast/}} und Frama-C\footnote{\url{http://frama-c.com}} in Kombination
mit der Annotationssprache ACSL (ANSI/ISO C Specification Langage)\footnote{\url{http://frama-c.com/acsl.html}} 
Der Fokus liegt dabei auf einfachen Algorithmen, die mit Arrays arbeiten ohne sie zu verändern.
Dabei wird die Lesbarkeit, Ausdrucksstärke sowie der notwendige manuelle Aufwand der Verifikation betrachtet.
Für den praktischen Einsatz wichtige Faktoren werden ebenso untersucht. Das sind insbesondere die verfügbare 
Dokumentation, die Integration in den Entwicklungsprozess und die Geschwindigkeit der Werkzeuge.
\newline
\newline
Als durchgängiges Beispiel dient in dieser Arbeit der standardisierte mismatch-Algorithmus\footnote{siehe
\url{http://msdn.microsoft.com/de-de/library/f0bsxbk9.aspx}} aus
der C++ Standard-Bibliothek. Die Grundlage für spätere Formalisierungen ist die folgende vereinfachte\footnote{es
wird auf Iteratoren und eine selbst definierte Vergleichsfunktion verzichtet} Signatur und 
die dazugehörige informelle Spezifikation:

\lstset{frame=none, numbers=none}    
\lstinputlisting[language=C]{codes/mismatch_signature.c}
\lstset{frame=single, numbers=left}

\noindent \emph{Vergleicht die Elemente der beiden Arrays beginnend bei 0 bis einschließlich size - 1 und gibt den
Index der ersten Elemente zurück, die sich unterscheiden. Sind alle Elemente gleich, gibt der Algorithmus
size - die Länge der Liste - zurück.}
\newline
\newline
Auf Grund des einfachen Beispiels hat der Vergleich nur eine begrenzte Aussagekraft, denn es werden nicht alle
Spracheigenschaften von VeriFast bzw. ACSL gezeigt und verglichen. Zudem kommt hinzu, dass sich beide Werkzeuge
kontinuierlich - in ihre jeweils eigene Richtung - weiterentwickeln. Es wird darum immer Anwendungsfälle geben,
in denen ein Werkzeug dem anderen überlegen ist oder sogar alternativlos ist.


\section{VeriFast}
\label{sec:VeriFast}

Veri

KU Löwen university, by bart jacobs


for C and java


functional correctness


seperation logic - angepasste hoare logic


malloc/free - keien speicherlöscher


multithreading .. z.B. lock free datenstrukturen

kein eigener name für die sprache, darum wird immer nur von VeriFast gesprochen


industrieeinsatz

\section{Frama-C}
\label{acsl-und-frama-c}

ansi/iso-c specification language


zusammen mit VeriFast teil des STANCE projekts

\section{Zielgruppe}
\label{sec:zielgruppe}

Für das Verstehen der Arbeit sollte der Leser die Grundlagen der Programmiersprache C beherrschen.
sowie theoretische Kenntnisse zur Aussagen- und Prädikatenlogik besitzen.
Konkrete Erfahrungen mit ACSL und Frama-C sind hilfreich, da beim Vergleich der Werkzeuge an vielen
Stellen VeriFast detaillierter als ACSL erläutert wird. Die Arbeit ist darum auch als Einstieg in VeriFast
für ACSL-Nutzer geeignet und ausgelegt.

 

