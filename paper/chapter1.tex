\chapter{Einleitung}

\section{Aufgabenstellung und Durchführung}
\label{sec:aufgabenstellung}
Diese Arbeit vergleicht die statischen Verifikationswerkzeuge Verifast und ACSL in Kombination mit Frama-C. 
Der Fokus liegt dabei auf einfachen Algorithmen, die mit Arrays arbeiten ohne sie zu verändern.
Dabei wird die Lesbarkeit, Ausdrucksstärke sowie der notwendige manuelle Aufwand der Verifikation betrachtet.
Für den praktischen Einsatz wichtige Faktoren werden ebenso untersucht. Das sind insbesondere die verfügbare 
Dokumentation, die dazugehörigen Tools sowie die Lernkurve beim Einsatz der Werkzeuge.

Als durchgängiges Beispiel dient in dieser Arbeit der standardisierte mismatch-Algorithmus aus
der C++ Standard-Bibliothek. In leicht vereinfachter Form sieht dessen Signatur wie folgt aus:


\lstinputlisting[language=C]{codes/mismatch_signature.c}

Als Grundlage für folgende Formalisierungen verwenden wir folgende informelle Spezifikation:

\todo{Erwähnen, dass return == size, wenn alles identisch?}
\emph{Vergleicht die Elemente der beiden Arrays beginnend mit 0 bis einschließlich size - 1 und gibt den
Index der ersten Elemente zurück, die sich unterscheiden.}

Auf Grund des einfachen Beispiels hat der Vergleich nur eine begrenzte Aussagekraft, denn es werden nicht alle
Spracheigenschaften von Verifast bzw. ACSL gezeigt und verglichen. Zudem kommt hinzu, dass sich beide Werkzeuge
kontinuierlich - in ihre jeweils eigene Richtung - weiterentwickeln. Es wird also immer Anwendungsfälle geben,
in denen ein Werkzeug dem anderen überlegen ist oder sogar alternativlos ist.



\section{Verifast}
\label{sec:verifast}

KU Löwen university, by bart jacobs


for C and java


functional correctness


seperation logic - angepasste hoare logic


malloc/free - keien speicherlöscher


multithreading .. z.B. lock free datenstrukturen


industrieeinsatz

\section{ACSL und Frama-C}
\label{acsl-und-frama-c}

ansi/iso-c specification language


zusammen mit verifast teil des STANCE projekts

\section{Zielgruppe}
\label{sec:zielgruppe}

Für das Verstehen der Arbeit sollte der Leser mindestens die Grundlagen der Programmiersprache C beherrschen.
Theoretische Kenntnisse zur Aussagen- und Prädikatenlogik sind ebenfalls sehr hilfreich.

Idealerweise hat der Leser bereits Erfahrung im Umgang mit ACSL und Frama-C und kann nach dem Lesen dieser Arbeit
ebenfalls mit Verifast arbeiten. 