\chapter{Einleitung}

\section{Aufgabenstellung und Durchführung}
\label{sec:aufgabenstellung}
Diese Arbeit vergleicht die Verifikationswerkzeuge VeriFast\footnote{
\url{http://people.cs.kuleuven.be/~bart.jacobs/VeriFast/}} und Frama-C\footnote{\url{http://frama-c.com}}.
Der Fokus liegt auf einfachen Algorithmen, die mit Arrays arbeiten, ohne diese zu verändern.
Dabei werden die verwendeten Sprachen mit ihren jeweiligen Konzepten betrachtet und verglichen. Außerdem
wird die Lesbarkeit bewertet und die Arbeit mit den Werkzeugen untersucht. 

Schlussendlich wird die daraus resultierende Lernkurve bei der Verfizierung beschrieben und geschlussfolgert
welches Werkzeug sich am ehesten für die beschriebenen Algorithmen eignet.

Als durchgängiges Beispiel dient in dieser Arbeit der standardisierte mismatch-Algorithmus\footnote{siehe
\url{http://msdn.microsoft.com/de-de/library/f0bsxbk9.aspx}} aus
der C++ Standard-Bibliothek. Die Grundlage für spätere Formalisierungen ist die folgende vereinfachte\footnote{es
wird auf Iteratoren und eine selbst definierte Vergleichsfunktion verzichtet} Signatur und 
die dazugehörige informelle Spezifikation:

\lstset{frame=none}    
\lstinputlisting[language=C]{codes/mismatch_signature.c}
\lstset{frame=single}

\noindent \emph{Vergleicht die Elemente der beiden Arrays beginnend bei 0 bis einschließlich size - 1 und gibt den
Index der ersten Elemente zurück, die sich unterscheiden. Sind alle Elemente gleich, gibt der Algorithmus
size - die Länge der Liste - zurück.}


\section{Frama-C}
\label{acsl-und-frama-c}

Frama-C ist eine modulare Werkzeugsammlung für die Analyse von C-Programmen entwickelt von CEA\footnote{\url{www.cear.fr}}
(französisches Kommissariat für Atomenergie und alternative Energien) sowie Inria\footnote{\url{www.inria.fr}} 
(französisches Forschungsinstitut für Informatik). Es beinhaltet viele verschiedene Plugins mit denen der C-Code
analysiert, transformiert oder verifiziert werden kann. 

In dieser Arbeit wird ausschließlich das WP-Plugin\footnote{WP steht für Weakest Precondition, siehe
\url{http://frama-c.com/wp.html}} für die deduktive Verifikation betrachtet. Um es zu verwenden, müssen die zu beweisenden 
Eigenschaften als ACSL-Annotationen (ANSI/ISO C Specification Langage)\footnote{\url{http://frama-c.com/acsl.html}} zum Quellcode hinzugefügt werden, woraus Frama-C dann die 
Beweispflichten (engl. proof obligations) in Form von prädikatenlogischen Formeln generiert. Angebundene automatische oder 
interaktive Beweiser wie Alt-Ergo\footnote{\url{http://alt-ergo.lri.fr/}} oder Coq\footnote{\url{http://coq.inria.fr/}}
versuchen diese dann zu lösen.

Die Verifikation basiert auf der Hoare-Logik, die 1969 von C.A.R. Hoare eingeführt wurde\cite{hoare}. Dabei
handelt es sich um ein formales System aus verschiedenen logischen Regeln, mit denen die Korrektheit von
Computerprogrammen bewiesen werden kann.


\section{VeriFast}
\label{sec:VeriFast}

VeriFast ist ein statisches Verifizierungswerkzeug für Java und C, entwickelt an der KU Leuven in Belgien.
Wie Frama-C unterstützt es die Verifikation von Spezifikationen, die in den Quellcode annotiert wurden. Die verwendete
Annotationssprache hat keinen eigenen Namen, darum ist mit \glqq VeriFast\grqq{} abhängig vom Kontext entweder die Sprache oder das
Werkzeug als solches gemeint.

Die Annotationen werden nicht über deduktive Verifikation, sondern mittels symbolischer Ausführung bewiesen. Das Austauschen
des Beweisers so wie in Frama-C ist außerdem nicht möglich\footnote{VeriFast nutzt den Z3-Beweiser von Microsoft Research,
\url{http://z3.codeplex.com/}}, da es sich um eine geschlossene Software handelt.

Das Programm versteht Quellcode-Annotationen geschrieben in der Separierungslogik, die 2002 von John C. Reynolds publiziert
wurde\cite{reynolds-2002}. Dabei handelt es sich um eine Erweiterung der Hoarelogik, die zusätzliche Aussagen über den Speicher
erlaubt.


\section{Zielgruppe}
\label{sec:zielgruppe}

Für das Verstehen der Arbeit sollte der Leser die Grundlagen der Programmiersprache C beherrschen.
sowie theoretische Kenntnisse zur Aussagen- und Prädikatenlogik besitzen.
Konkrete Erfahrungen mit ACSL und Frama-C sind hilfreich, da beim Vergleich der Werkzeuge an vielen
Stellen VeriFast detaillierter als ACSL erläutert wird. Die Arbeit ist darum auch als Einstieg in VeriFast
für ACSL-Nutzer geeignet und ausgelegt.

 

