\chapter{Vergleich der Werkzeuge}

In diesem Kapitel stehen die Werkzeuge im Fokus des Vergleichs. Neben der konkreten Arbeit mit der
Oberfläche wird auch die Integration in den Entwicklungsprozess betrachtet sowie die Lernkurve für
die erfolgreiche Verifizierung eigener Software.

\section{Arbeit mit den Werkzeugen}

Die Arbeit mit den beiden Werkzeugen unterscheidet sich stark voneinander, da die Systeme
neben dem unterschiedlichem Verifizierungsansatz auch eine andere Architektur aufweisen.
VeriFast setzt auf die symbolische Ausführung des Codes und kommt ohne ein modulares Konzept aus.
Die einzelnen Bestandteile der Software sind fest verbaut und können nicht ausgetauscht werden.
Frama-C hingegen ist aus mehreren Modulen bzw. Plugins aufgebaut, die an verschiedenen Stellen des
Verfizierungsprozesses zum Einsatz kommen. Das hat auch einen Einfluss auf die Möglichkeiten, welche
die entsprechenden Oberflächen anbieten können.

Das WP-Plugin von Frama-C konstruiert aus dem annotiertem Quellcode die Beweisziele und gibt diese an 
entsprechende installierte Beweiser weiter. Das hat den Vorteil, dass für einzelne Aussagen die Stärken
spezieller Beweiser genutzt werden könnten. Falls notwendig, kann sogar manuell nachgeholfen werden, in
dem man den interaktiven Coq-Beweiser nutzt.

In VeriFast besteht diese Möglichkeit nicht. Dafür findet der Beweisvorgang mit Hilfe der symbolischen
Ausführung wesentlich transparenter statt, denn jeder Schritt ist nachvollziehbar und es ist schnell ersichtlich,
wieso ein Beweis fehl schlägt. In Frama-C hingegen verhalten sich die Beweiser wie eine Black Box, da
bei nicht erfolgreicher Verifizierung unklar ist wieso der Beweis einer einzelnen Annotation
fehlgeschlagen ist.

Die Arbeit an den Annotationen fällt mit VeriFast wesentlich leichter als mit Frama-C, da die
Oberfläche das direkte Bearbeiten der Quellen erlaubt. Zudem findet die Verifizierung in VeriFast
schneller statt. Das liegt daran, dass der Beweisvorgang selber nicht sehr rechenaufwendig und damit auch schneller\cite[Kap. 3]{jac10-1}
ist. Der Preis dafür ist die Notwendigkeit weitere detaillierte Annotationen zu ergänzen, die in ACSL nicht erforderlich
sind.


\section{Integration in den Entwicklungsprozess}

Für beide Werkzeuge gilt, dass sie nur alleine lauffähig sind und nicht im Rahmen
einer Entwicklungsumgebung wie Eclipse\footnote{\url{www.eclipse.org}} genutzt werden können. 
Für die Verifizierung muss also der Arbeitsfluss unterbrochen werden. Denn auch wenn die Verifizierung
nicht vom Entwickler durchgeführt werden sollte, wäre es hilfreich in einer einheitlichen integrierten Umgebung
den Code auszuführen, zu testen, zu debuggen und verifizieren zu können.

Die Integration in einen automatisieren Build-Prozess ist mit beiden Werkzeugen möglich, da die Verifizierung nicht nur
über die Oberfläche, sondern auch per Kommandozeile möglich ist. Damit ist auch die Einbettung in eine
Continous-Integration-Umgebung\footnote{\url{http://de.wikipedia.org/wiki/Kontinuierliche_Integration}} denkbar,
in der die Verifizierung automatisiert nach jeder Änderung erneut ausgeführt wird, um Regressionsfehler zu entdecken.


\section{Lernkurve}

Die Lernkurve bei der Arbeit mit den Werkzeugen sieht unterschiedlich aus, was bereits aus der Gliederung dieser
Arbeit ersichtlich wird. Im Kapitel drei wurde der Vergleich der Kontrakte mit ACSL eingeleitet, da zur
Spezifizierung von \texttt{equal} keine fortgeschrittenen Konzepte benötigt wurden. Die Variante mit VeriFast
hingegen benötigt induktive Listen, rekursive Prädikate sowie Mustererkennung. Selbst für ein so einfaches Beispiel
wird in VeriFast also gleich ein Großteil der Sprachmittel benötigt. Frama-C ist da einsteigerfreundlicher, so
wie es die Website mit dem Spruch \glqq Simple things should be simple, complex things should be possible\grqq{} verspricht.

Dies setzt sich bei der Verifizierung der Implementierung fort. Wie in Kapitel vier ersichtlich, benötigt VeriFast
zusätzlich zu einer Schleifeninvarianten einen Lemma-Aufruf, um \texttt{mismatch} zu verifizieren. ACSL kommt ohne aus,
da die angeschlossenen Beweiser mächtiger sind. Auch in diesem Fall ist es also bei der Arbeit mit Frama-C nicht
notwendig weitergehende Konzepte wie Axiome bzw. Lemmata zu verstehen und einzusetzen.

Für die in dieser Arbeit gezeigten Algorithmen ist also die Verizierung mit Frama-C schneller erlernbar. Für komplizierte
Anwendungen hingegen ist zu vermuten, dass sich der Lernaufwand recht schnell angleicht, da dann in beiden Sprachen
alle verfügbaren Sprachkonstrukte verstanden und genutzt werden müssen. Tatsächlich sollte die Lernkurve bei der Nutzung
von VeriFast sogar schneller an ihre Grenze stoßen als bei Frama-C. Das liegt daran, dass die Verifizierung mit dem WP-Plugin
von Frama-C nicht bei der Anwendung von ACSL enden muss. Es kann durchaus notwendig sein einen interaktiven Beweiser wie Coq
zu verstehen und zu nutzen. Da ist VeriFast mit seinem transparenten Beweisvorgang im Vorteil, insbesondere
dank der einfach zu verstehenden Oberfläche.
